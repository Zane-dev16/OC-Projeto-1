\documentclass{article}
\usepackage[utf8]{inputenc}
\usepackage[portuguese]{babel}
\usepackage{amsmath}
\usepackage{listings}
\usepackage{color}
\usepackage{titling}  % Pacote para customização do título

% Ajustar espaçamento do título
\setlength{\droptitle}{-3cm}  % Aumenta o valor negativo para subir mais

% Definir avanço de parágrafo
\setlength{\parindent}{12pt}  % Define um avanço de 12pt nos parágrafos

\definecolor{codegreen}{rgb}{0,0.6,0}
\definecolor{codegray}{rgb}{0.5,0.5,0.5}
\definecolor{codepurple}{rgb}{0.58,0,0.82}
\definecolor{backcolour}{rgb}{0.95,0.95,0.92}

\lstdefinestyle{mystyle}{
    backgroundcolor=\color{backcolour},   
    commentstyle=\color{codegreen},
    keywordstyle=\color{magenta},
    numberstyle=\tiny\color{codegray},
    stringstyle=\color{codepurple},
    basicstyle=\ttfamily\footnotesize,
    breakatwhitespace=false,         
    breaklines=true,                 
    captionpos=b,                    
    keepspaces=true,                 
    numbers=left,                    
    numbersep=5pt,                  
    showspaces=false,                
    showstringspaces=false,
    showtabs=false,                  
    tabsize=1
}

\lstset{style=mystyle}

\title{Relatório do Projeto de Simulador de Cache}
\author{Duarte Ramires, David Quintino e Irell Zane}
\date{\today}

\begin{document}

\maketitle

\section{Introdução}
Este relatório descreve a implementação das três tarefas do primeiro projeto de OC, abordando o seguinte:

\begin{itemize}
    \item \textbf{Cache L1 (Um nível com Mapeamento Direto)}
    \item \textbf{Cache L2 (Dois níveis com Mapeamento Direto)}
    \item \textbf{Cache L2 (2-Way Set Associative)}
\end{itemize}

\section{Cache L1 de Mapeamento Direto}
\subsection{Cache de múltiplas linhas}
O código base fornecia uma cache que armazenava apenas uma linha. A estrutura da cache foi alterada para ser uma \textit{struct} contendo várias linhas. 

\subsection{Interpretação do endereço}
Foi criada uma lista de funções para obter as diferentes partes do endereço:
\begin{itemize}
    \item \texttt{getIndex} para obter o índice de um bloco na cache.
    \item \texttt{getBlockOffset} para obter o índice da palavra dentro do bloco na cache.
    \item \texttt{getMemAddress} para obter o endereço da memória onde o bloco de dados está armazenado.
\end{itemize}

O código base foi atualizado para utilizar estas funções. Assim, o argumento do endereço consegue ser interpretado corretamente para obter a linha correspondente na cache e realizar as operações corretas.

\subsection{Write-back policy}
A última alteração realizada trata dos \textit{cache misses} enquanto a linha da cache contém dados com um \textit{dirty flag}. A função \texttt{getMemAddressFromCacheInfo} obtém o endereço da DRAM que corresponde aos dados \textit{dirty} na cache. Com isso, é possível fazer o \textit{write-back} para o endereço correto na DRAM antes de atualizar a linha/bloco na cache, seguindo a \textit{write-back policy}.

\section{Cache L2 - 2 níveis com Mapeamento Direto}

\subsection{Função da Cache L2 como camada intermédia} O acesso à cache L1 é feito primeiro, e, quando não encontra o bloco de dados solicitado (cache miss), a função \texttt{accessL2Cache} é chamada para procurar esse bloco em L2. Caso o bloco não esteja presente (cache miss duplo), a busca continua na DRAM. Após um cache miss em L2, o bloco recuperado da DRAM é armazenado tanto em L2 como em L1, utilizando a função \texttt{memcpy} para otimizar o tempo de acesso em futuras operações. 

\subsection{Estrutura da Cache L2} A cache L2 é semelhante à cache L1, mas possui uma maior capacidade. As operações de leitura e escrita na cache L2 seguem uma política de mapeamento direto, implementada nas mesmas funções que em L1. A inicialização da cache L2 é tratada pela função \texttt{initCaches}, que prepara as linhas de cache para acesso.

\subsection{Política de Write-back} L2 implementa uma \textit{write-back policy} que abrange tanto a cache L1 como a DRAM. Quando uma linha marcada como \textit{dirty} na cache L2 precisa de ser substituída, a mesma lógica da cache de um nível é aplicada. Da mesma forma, se uma linha na cache L1 precisar de ser substituída e estiver marcada como \textit{dirty}, os dados serão escritos na L2 utilizando a função \texttt{accessL2Cache} antes da substituição. Assim, as alterações feitas nas caches não são perdidas, e tanto a L2 como a DRAM terão os dados mais recentes.

\end{document}
